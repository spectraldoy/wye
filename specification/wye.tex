% --- Preamble ---
\documentclass[a4paper, 12pt]{article}

% --- Packages ---
\usepackage{microtype}
\usepackage{amsmath}
\usepackage{amsfonts}
\usepackage{amsthm}
\usepackage{enumitem}
\usepackage[english]{babel}
\usepackage[titletoc, title]{appendix}
\usepackage[bottom]{footmisc}
\usepackage{listings}
\usepackage[utf8]{inputenc}
\usepackage[T1]{fontenc}
\usepackage{inconsolata}
\usepackage{minted}

\usepackage{setspace}
\usepackage[a4paper, margin=1.15in]{geometry}

\usepackage[square, numbers, comma]{natbib}

\usepackage{xcolor}
\usepackage{hyperref}
\hypersetup{
    colorlinks=true,
    linkcolor=black,
    filecolor=magenta,      
    urlcolor=blue,
    citecolor=black
}
\urlstyle{same}

% set up Wye lstlistings
\definecolor{darkgreen}{rgb}{0.125, 0.73828125, 0.421875}

\lstdefinelanguage{wye}{
  keywords={let, type, with, if, then, else, match, print, error},
  keywordstyle=\color{darkgreen}\bfseries,
  keywords=[2]{int, string, float},
  keywordstyle=[2]\color{blue},
  keywords=[3]{struct, record, nom, impl, sig}
  keywordstyle=[3]\color{orange},
  comment=[l]{\%},
  commentstyle=\color{gray},
  identifierstyle=\color{black},
  stringstyle=\color{red},
  morestring=[b]"
}

\lstset{
   language=wye,
   extendedchars=true,
   basicstyle=\ttfamily\footnotesize,
   showstringspaces=false,
   showspaces=false,
   tabsize=2,
   breaklines=true,
   showtabs=false
}

% newcommands
\newcommand{\bits}{\{0, 1\}}
\newcommand{\sepbar}{\: | \:}	% separated bar
\newcommand{\substo}{\quad\rightarrow\quad}
\newcommand{\uscore}{\underline{\hspace{0.3cm}}}
\newcommand{\version}{2.1.0}
\newcommand{\dedrule}[2]{\begin{align*}\frac{$1}{$2}\end{align*}}
\newcommand{\free}{\text{free}}
\renewcommand{\tt}{\texttt}
\newcommand{\la}{\langle}
\newcommand{\ra}{\rangle}

% --- Document ---
\iffalse
Abstract: origin of Wye, what it is
1 Introduction: 
- Basic description of language features and type system

2 Syntax
- Grammar

3 Semantics
one by one

4 Type system
Hindley Milner but with some cool stuff

5 Intermediate representation
- using LLVM, it will be compilable

Appendix:
- Features that Wye does not have
\fi

\allowdisplaybreaks
\begin{document}

% --- Title ---
\title{
\textsc{Wye: A Compiled, Functional, Object-Oriented Language} \\
\vspace{2ex}
\large{\textsc{Version \version}}\\
\vspace{2ex}
}
\author{\normalsize\textsc{Aditya Gomatam} \\ 
\normalsize{\today\vspace{2ex}}}
\date{}
\maketitle

\setstretch{1.15}

% --- Abstract --- 
\begin{abstract}
Wye is a statically-, strongly-, and structurally-typed, functional, object-oriented language,
heavily inspired by Haskell, Rust, and OCaml. Wye recognizes there is a conceptual burden
imposed the type system in many languages that can inhibit the development of programs.
In response to these issues, as well to explore the beauty of type theory in both its mathematical and
practical incarnations, Wye centers around structural object typing and type inference.
Nevertheless, Wye retains a firm basis in functional programming,
indeed alluding in its name to the upside-down \texttt{y} of the lambda calculus.

This document compiles the full specification of the Wye language: its syntax, semantics,
type system, and intermediate representation.
\end{abstract}

% WYE: only struct objects can hold state. You cannot define methods on anon records
% anon records should not be allowed to change state with methods. If you want
% to store state, then declare a struct

% List construction should be list : value -> and it should look like arrays, not linked lists
% Holding types and values in interfaces is fine?
% Also list construction should have different syntax, and the . operator shouldn't be so
% overloaded. The :: operator could be more useful to for things other than list construction

% TODO: also change the syntax for implementing an interface
% impl 'Ordered X | impl Ordered'X
% There should also be a builtin Indexable interface - lists should be indexable
% or GetItem or something like htat

% <: for subtype syntax

% for the minimal release, we can get rid of methods on records, and consider having
% nominality of record matching as per-argument, or with {< } and {> } constructs.
% This makes structural subtyping much more expressive
% Also methods and values may need to be stored in a specific order so that their
% compilation is deterministic. However, they will never be deleted, with the current
% syntax we have in Wye. Maybe an ordered map can be optimized with this info.


% --- Body ---
\section{Introduction}
Wye is a statically-, strongly-, and structurally-typed functional, object-oriented language,
heavily inspired by Haskell, Rust, and OCaml. In recognition that the type system can have
a drastic impact on the tendency of programmers to write maintainable or unintelligible programs,
Wye invests a great deal in building a safe, expressive, yet straightforward type system.

A specific point of interest is in facilitating design work through the language. Design work is inherently
unaware of its final form; however, explicitly-typed languages like C++ require that you know
the types of your entities before you've completely fleshed out your thoughts. On the other hand,
dynamically-typed languages like JavaScript and Python can be especially convenient during the
design phase, affording an open space to play; but, they readily become unintelligible,
unmaintainable, messy, and sorts of other undesirable adjectives.

Wye seeks to strike a balance between these two extremes. How can a language be designed so
that it both supports design work, while also encouraging maintainability and intelligibility? I currently
do not know the answer. But a hypothesis is examined in this work -- extend basic functional programming
(which is generally very academic, but which has perfect type inference) with structural object types
and interface programming with bounds (features present in most industry-grade object-oriented
languages).

\section{Syntax}
A Wye program is stored in a \tt{.wye} file and is, in its most abstract
view, simply a sequence of Wye statements.

Written below in Backus-Naur form is roughly the Wye grammar. The start symbol is
$Program$. Nonterminals start with a capital letter. Terminals are written in
\tt{this font}. $pat*$ denotes the Kleene star: zero or more repetitions of
the pattern $pat$. $pat+$ is a shorthand for $pat\: pat*$. $pat?$ denotes that
$pat$ may or may not occur. $( pat_1\:...\: pat_n )$ groups $pat_1, ..., pat_n$
into a new pattern in which each $pat_i$ is to occur in the specified sequence.
$pat1 | pat2$ means that exactly one of $pat1$ or $pat2$ may occur. Beware that
\tt{|} and $|$ are different -- the first is a Wye token. $\la$These
brackets$\ra$ are used to annotate certain grammar rules.

\subsection{Grammar}
\subsubsection{Top-level non-terminals}
\begin{align*}
Program \substo& Statement+\\
Statement \substo& Expr \sepbar EnumDecl \sepbar StructDecl\\
&\sepbar SigDecl \sepbar ImplBlock
\end{align*}
\subsubsection{Expressions}
\begin{align*}
Expr \substo& IntLiteral \sepbar FloatLiteral \sepbar StringLiteral\\
&\sepbar List \sepbar Tuple \sepbar AnonRecord\\
&\sepbar Id \sepbar BuiltinOp\\
&\sepbar \tt{print} \sepbar \tt{error}\\
&\sepbar TypeId\tt{.}TypeId \: (\tt{with} \: Expr) \: \la\text{enum variant}\ra \\
&\sepbar Expr \: Expr \: \la\text{function application}\ra \\
&\sepbar Expr \: BuiltinOp \: Expr \: \la\text{reserved binary op}\ra \\
&\sepbar \tt{match}\: Expr\: \tt{\symbol{92}n} \:(Pat \: \tt{=>} \: Expr\:\tt{\symbol{92}n})* Pat \: \tt{=>} \: Expr\:\tt{end}\\
&\sepbar \tt{\symbol{92}}\:Id\: \tt{->}\: Expr\:\la\text{lambda expression}\ra \\
&\sepbar Id\tt{.}Id \: \la\text{member access}\ra\\
&\sepbar Id\tt{\#}Id \: \la\text{method access}\ra\\
&\sepbar \tt{(}\: Expr \: \tt{)}\\
&\sepbar LetExpr \: (\tt{in}\: Expr)?\\
&\sepbar AttrSet\\
List \substo& \tt{[}\: (Expr \: \tt{,})*\: Expr\: \tt{]} \sepbar \tt{[]}\\
Tuple \substo& \tt{(}\: (Expr \: \tt{,})+\: Expr\: \tt{)}\\
AnonRecord \substo& \tt{\{} (Id\tt{:} Expr\tt{,})+ (Id\tt{:} Expr\: \tt{,}?) \tt{\}}\\
BuiltinOp \substo& \tt{+}\sepbar\tt{-}\sepbar\tt{*}\sepbar\tt{/}\sepbar\tt{//}\sepbar\tt{::}\sepbar\tt{<}\sepbar\tt{<=}\sepbar\tt{>}\sepbar\tt{>=}\sepbar\tt{==}\sepbar\tt{!=}\\
Pat \substo& \uscore \sepbar IntLiteral\sepbar FloatLiteral\sepbar StringLiteral\sepbar Id \\
& \sepbar Id\:\tt{::}\: Id\:\la\text{head::tail list destructuring}\ra\\
& \sepbar TypeId \: (\tt{with}\: Pat)?\\
& \sepbar \tt{[} \: ( Pat\:\tt{,})*\: Pat \: \tt{]} \sepbar  \tt{[]}\\
& \sepbar \tt{(} \: ( Pat\:\tt{,})+\: Pat\: \tt{)}\\
& \sepbar \tt{\{} \: (Id\tt{:}Pat\tt{,})*\: (Id\tt{:}Pat\: \tt{,}?) (\tt{,}\:\uscore)?\: \tt{\}}\\
& \sepbar \sim\:Pat \:\la\text{pattern complement}\ra\\
& \sepbar Pat\: (\tt{|}\: Pat)+\:\la\text{pattern union}\ra\\
& \sepbar Pat\:\tt{if}\: Expr\:\la\text{guarded pattern}\ra\\
& \sepbar \tt{case}\: Expr\:\la\text{check boolean }Expr\ra\\
LetExpr \substo& \tt{let}\: Id \:(\tt{:} Type)? \: \tt{=} \: Expr\\
&\sepbar \tt{let}\: Id \: Id+ \: \tt{=} \: Expr\\
&\sepbar \tt{let}\: Id \: (\: \tt{(} Id\tt{:} Type \tt{)}\: \tt{->} \:)+ Type \: \tt{=}\: Expr\\
AttrSet \substo& \tt{set}\: Id\tt{.}Id\: \tt{=} \:Expr\\
\end{align*}
\subsubsection{Records and Signatures}
\begin{align*}
EnumDecl \substo& \tt{enum}\: TypeId \: (\tt{'}\: TypeId?\: Id)* \: \texttt{=} \: (TypeId \: (\texttt{with}\: Type)?)\\
&(\texttt{|}\:TypeId \: (\texttt{with}\: Type)?)*\\
StructDecl \substo& \tt{struct}\: Id \: (\tt{'}\: TypeId?\: Id)*\: \tt{\symbol{92}n} \: (Id\tt{:} Type\: \tt{\symbol{92}n})+\: \tt{end}\\
SigDecl \substo& \tt{sig}\: Id\: ( \tt{implies} \: (Id \: \tt{+})*\: Id)? \:  \tt{\symbol{92}n}\\
&(ValDecl \sepbar MethodImpl \sepbar MethodDecl)+\: \tt{end}\\
ImplBlock \substo& \tt{impl}\: Id (\tt{'}\: TypeId?\: Id)* \: \tt{:} Id (\tt{'}\: TypeId?\: Id)* \: \tt{\symbol{92}n}\\
&(AttrSet \sepbar MethodImpl)+\: \tt{end} \\
ValDecl \substo& \tt{val} \: Id\: \tt{:}\: Type \\
MethodDecl \substo& \tt{method} \: Id\: \tt{:} \: Type\\\
MethodImpl \substo& \tt{method} \: Id\: Id*\: \tt{=}\: Expr\\
&\sepbar \tt{method}\: Id \: (\: \tt{(} Id\tt{:} Type \tt{)}\: \tt{->} \:)+ Type \: \tt{=}\: Expr\\
\end{align*}
\subsubsection{Types and type expressions}
\begin{align*}
Type \substo& \tt{int} \sepbar \tt{float} \sepbar \tt{string}\\
& \sepbar TypeId\: (Type)*\\
& \sepbar \tt{[}\:Type\:\tt{]}\\
& \sepbar \tt{(} \: (Type\: \tt{,} )+\: Type\: \tt{)}\\
& \sepbar TypeId? \: \tt{\{}\: \tt{method}?\: Id \: \tt{:}\: Type \: \tt{\}}\\
& \sepbar \texttt{'}\:TypeId?\: Id \: \la\text{(bounded) type variable}\ra\\
& \sepbar Type \: \tt{-> }Type\:\la\text{function type}\ra\\
\end{align*}
\subsubsection{Notes on the grammar}
The exact substitution rules for the $Literal$ patterns are omitted to avoid
boring the reader. If you are reading this, you know what a float is. Similarly,
the rules for $Id$ and $TypeId$ are omitted, though they are the same: a string
of digit, lowercase, underscore, or uppercase ASCII characters, not starting
with a digit. Note that while Wye uses ASCII as its character set for
identifiers and type names, strings may contain Unicode characters.

$Id$s and $TypeId$s should not be any of the builtin keywords such as
\tt{int}, \tt{float}, \tt{string}, \tt{print}, \tt{match},
\tt{with}, \tt{requires}, and so on. $Id$s and $TypeId$s must not conflict
with each other within their scope.

Notice that \tt{bool} is not a builtin type. This is because the
$TypeDeclaration$ system of Wye is used to define it in the Wye prelude.

One-line Wye comments begin with \tt{\%} and mark all following text until
the next carriage return or newline as whitespace. Multiline Wye comments begin
with \tt{[\%} and end with \tt{\%]}.

In Wye, the application of functions is always written in postfix notation,
except for certain reserved binary operators (such as \tt{+} and
\tt{::}) that may be written in infix notation. These binary operations are,
under the hood, translated into postfix notation.

\section{Semantics}

A Wye program is parsed, as noted earlier, as a sequence of Wye statements. A Wye
program is then executed top-to-bottom. The semantics of each line of the grammar are
described in the subsections of this section.

\subsection{Expressions}

\subsubsection{Patterns}

\subsection{Enums}

\subsection{Structures, interfaces and implementations}

Methods that define shared code cannot be overridden. If you want to override them, that indicates you
actually want to implement a different interface than what is being shared. That is, your code either
should not be shared, or the implementation you want to provide in the override should have a
different name.

\section{Future plans}

\subsection{Type Aliases}
Similar to TypeScript, it would be nice if the user could do something like the
following to create an abbreviation for a type:
\begin{lstlisting}{wye}
; just a tuple-type of length 2
type Pair 'a 'b = ('a, 'b);
\end{lstlisting}

\subsection{Dependent types}
Introduce a \texttt{'a val} type that can be used to compile-time create conditions
for typing dependent on that value. The value should not be computed via an expression
but should be listed explicitly - there will have to be a notion of an "explicit" expression.


\subsection{List Comprehension}
List comprehension should not be too hard to implement compared to the rest of
this project, but it will require some additions to the grammar, and is not
a strong focus of the initial versions of Wye.

\subsection{Modules}

\subsection{Intersection and Union types}

\subsection{User Input}
In Wye version \version{}, users may only output to the screen. Taking user
input is slightly more complicated and is thus deferred to a future version of
Wye.

\pagebreak
\section{Acknowledgements}
The syntax and semantics of Wye are based on Rust, Haskell, and OCaml. I thank
the developers of these languages for making it much easier for me to learn
about and write a compiler. I thank Jens Palsberg and Carey Nachenberg of UCLA
for teaching me the fundamentals of compiler implementation and programming
languages.

% --- Bibliography ---
\pagebreak
% change bibliography style if you want
\bibliographystyle{ieeetr}
\bibliography{references}

% --- Appendices ---
% \pagebreak
% \begin{appendices}
% \section{Thoughtfully named Appendix}
% \end{appendices}

\end{document}